\documentclass{article}
\usepackage[utf8]{inputenc}
\usepackage[spanish]{babel}
\usepackage{listings}
\usepackage{graphicx}
\graphicspath{ {images/} }
\usepackage{cite}

\begin{document}

\begin{titlepage}
    \begin{center}
        \vspace*{1cm}
            
        \Huge
        \textbf{Taller Memoria}
            
        \vspace{0.5cm}
        \LARGE
        Informática 2
            
        \vspace{1.5cm}
            
        \textbf{John Jairo Uribe Giraldo}
        
            
        \vfill
            
        \vspace{0.8cm}
            
        \Large
        Despartamento de Ingeniería Electrónica y Telecomunicaciones\\
        Universidad de Antioquia\\
        Medellín\\
        Septiembre de 2020
            
    \end{center}
\end{titlepage}

\tableofcontents

\section{Sección introductoria}
Esta es el primer taller de la materia de Informática 2, en el se dará respuesta a 4 preguntas planteadas, la primera acerca de la memoria del computador; la segunda pregunta se refiere a los tipos de memoria conocidos y una breve descripción; la tercera a la manera como se gestiona la memoria y por último acerca de la velocidad de los diferentes tipos de memoria.

\section{Sección de contenido} \label{contenido}

Defina que es la memoria del computador:
Es el componente del computador que  que apoya los procesos de almacenamiento de información, e incluso permite ejecutar procesos temporales de los programas.Un computador sin memoria, no podria arrancar.

Mencione los tipos de memoria que conoce y haga una pequeña descripción de cada tipo:

Memoria RAM(Random Access Memory):Memoria de Acceso Aleatorio, como su nombre lo indica permite almacenar informacion de manera temporal, es decir que cuando se reinicia el pc o hay ausencia de fluido eléctrico, lo que hay en la memoria ram se borra. 
Existen dos clases de memoria ram, sdram y dram, estática y dináica respectivamente

Memoria ROM (Ready Only Memory): Memoria de solo lectura, almacena informacion importante del sistema y de algunos programas.

Memoria Cache, es un tipo de memoria Ram, pero de rapido acceso, alta velocidad.Sirve de soporte al procesador almacenando instrucciones y datos de los cuales el procesador debe contar en cualquier momento. 

Describa la manera como se gestiona la memoria en un computador:
La memoria en el computador se debe administrar eficientemente, algunos programas necesitan unos campos de memoria para funcionar eficientemente, y cuando ya no se requiere la gestion de memoria libera estos espacios para futuros programas.
Todo este registro lo lleva el administrador de memoria, proporcionando protección y uso compartido, debe facilitar un espacio para cada proceso.

Qué hace que una memoria sea más rápida que otra?:
La velocidad se debe a tres factores importantes: a la velocidad del bus, a la frecuencia de reloj que trabaja el bus de datos, y a la cantidad de bits que se transfieren por este bus. Con el paso del tiempo la tecnologia ha ido mejorando estas características dentro de la arquitectura de los computadores o más específicamente de las placas madre o Main Board.

Por qué es importante?:Es mportante porque la velocidad me esta definiendo la tasa de transferencia de información en bits por segundo, y para que este proceso sea eficiente, las memorias deben ir avanzando según la arquitectura de los computadores actuales, especificamente los buses de datos.



\section{Conclusión} \label{conclulsion}

La tecnología esta avanzando a un ritmo exponencial, los sistemas electrónicos han evolucionado cambiando su arquitectura y por ende la memoria como dispositivo importante de los procesos de computo lo hace también, la memoria ha evolucionado, tanto la cache para el procesador, como la Rom y la Ram para permitir procesos más eficientes.

\bibliographystyle{IEEEtran}
\bibliography{references}

\end{document}
